\section{Complex Analysis Review}
Before we can go too deep in Kähler manifolds (and some Riemannian geometry in
general), we need to have a quick review of complex analysis.

\subsection{Holomorphic Functions}

\begin{definition}\label{def 3.1}
    Given an open set $\Omega \in \C$, a function $f: \Omega \to \C$ is \ten{analytic}
    if for all $z_0 \in \Omega$ there exists a ball of radius $\varepsilon>0$ about
    $z_0$ such that $f$ has a well-defined power series:
    \begin{equation*}
        f(z) = \sum_{n=0}^\infty a_n (z-z_0)^n, \quad \forall z \in B_\varepsilon
        (z_0)
    \end{equation*}
\end{definition}

Some refer to this definition as a \textbf{holomorphic} function but this is technically
incorrect. Analyticity is a local property defined about some $\varepsilon$-neighbordhood
whereas being holomorphic is a little more general. The differentiable property of
complex functions is defined only pointwise. A classical example of this difference
would be a function that is only differentiable about a line and thus not analytic. 
However, in complex analysis these two definitions (analytic and holomorphic) happen
to be equivalent so they are often used interchangeabley. The proof is not too difficult
but we omit it here. We will however establish the exact definition of holomorphic.

\begin{definition}\label{3.2}
    A function $f: \Omega \to \C$, expressed as $f(x,y) = u(x,y) + iv(x,y)$,
    is \ten{holomorphic} if it satisfies the Cauchy-Riemannan
    equations:
    \begin{equation*}
        \dfrac{\partial u}{\partial x} = \dfrac{\partial v}{\partial y}, \quad
        \dfrac{\partial u}{\partial y} = - \dfrac{\partial u}{\partial x}
    \end{equation*}
\end{definition}
In fact, we can generalize this to $\C^n$. Now let $f: \Omega \subseteq \C^n \to
\C^n$. This function is holomorphic on $\C^n$ if it satisfies the generalized 
Cauchy-Riemann equations:
\begin{equation*}
    \dfrac{\partial u}{\partial x_i} = \dfrac{\partial v}{\partial y_i}, \quad 
    \dfrac{\partial u}{\partial y_i} = - \dfrac{\partial v}{\partial x_i},\quad
    \text{for } i = 1, 2, \hdots , n
\end{equation*} 
In laymen's terms, $f$ is holomorphic in every ``direction'' or coordinate. This
gives us a perfect segue into how the complex Jacobian matrix for a holomorphic function
$f$ is related to and defined by the Cauchy-Riemann equations:
\begin{equation*}
    J(f)(z):= \bigg(\dfrac{\partial f_i}{\partial z_j}(z)\bigg), \ 1 \leq i \leq n, \
    1 \leq j \leq m
\end{equation*}
Recall from differential geometry that Jacobian matrices define transition maps.
The complex Jacobian matrices also play a part in complex manifolds as we will see
later.