\documentclass[12pt, letterpaper]{article}
\usepackage{inputenc, amsmath, amssymb, bm, mathrsfs, mathtools, hyperref, amsthm}
\usepackage[margin=.8 in]{geometry}
\usepackage{enumerate}
\usepackage{graphicx}
\usepackage[font=footnotesize,labelfont=bf]{caption}
\usepackage{subcaption}
\usepackage{titlesec}
\usepackage{setspace}
\setlength\parindent{0pt}

\newtheorem{theorem}{Theorem}[section]
\newtheorem{corollary}{Corollary}[theorem]
\newtheorem{lemma}[theorem]{Lemma}
\newtheorem{definition}{Definition}[section]
\newtheorem{example}{Example}[section]

% Will def need this later, should probably do this rn though :(
% \usepackage[backend=biber]{biblatex}
% \let\cite=\supercite
% \addbibresource{refs.bib}

\newcommand\myeq{\stackrel{\mathclap{\normalfont\mbox{def}}}{=}}
\newcommand{\R}{\mathbb{R}}
\newcommand{\C}{\mathbb{C}}
\newcommand{\Half}{\mathbb{H}}
\newcommand{\Z}{\mathbb{Z}}
\newcommand{\Q}{\mathbb{Q}}
\newcommand{\N}{\mathbb{N}}
\newcommand{\ten}[1]{\textnormal{\textbf{#1}}}


\title{Notes}
\author{Preston Malen}
\date{January 2024}

\begin{document}

\maketitle
\thispagestyle{empty}


\newpage
\thispagestyle{empty}
\tableofcontents


\newpage
\clearpage

\section*{Introduction}

\newpage
\section{Complex Analysis Review}
Before we can go too deep in Khäler manifolds (and some Riemannian geometry in
general), we need to have a quick review of (multivariate) complex analysis.

\subsection{Holomorphic Functions}

\begin{definition}\label{1.1}
    Given an open set $\Omega \in \C$, a function $f: \Omega \to \C$ is \ten{analytic}
    if for all $z_0 \in \Omega$ there exists a ball of radius $\varepsilon>0$ about
    $z_0$ such that $f$ has a well-defined power series:
    \begin{equation*}
        f(z) = \sum_{n=0}^\infty a_n (z-z_0)^n, \quad \forall z \in B_\varepsilon
        (z_0)
    \end{equation*}
\end{definition}

Some refer to this definition as a \textbf{holomorphic} function but this is technically
incorrect. Analyticity is a local property while being holomorphic is a global property.
However, in complex analysis these two definitions happen to be equivalent so they 
are often use interchangeabley. The proof is not too difficult but we omit it here.
We will however establish the exact definition of holomorphic.

\begin{definition}\label{1.2}
    A function $f: \Omega \to \C$, typically expressed as $f(x,y) = u(x,y) + iv(x,y)$,
    is \ten{holomorphic} if it satisfies the Cauchy-Riemannan
    equations:
    \begin{equation*}
        \dfrac{\partial u}{\partial x} = \dfrac{\partial v}{\partial y}, \quad
        \dfrac{\partial u}{\partial y} = - \dfrac{\partial u}{\partial x}
    \end{equation*}
\end{definition}
In fact, we can generalize this to $\C^n$. Now let $f: \Omega \subseteq \C^n \to
\C^n$. This function is holomorphic on $\C^n$ if it satisfies the generalized 
Cauchy-Riemann equations:
\begin{equation*}
    \dfrac{\partial u}{\partial x_i} = \dfrac{\partial v}{\partial y_i}, \quad 
    \dfrac{\partial u}{\partial y_i} = - \dfrac{\partial v}{\partial x_i},\quad
    \text{for } i = 1, 2, \hdots , n
\end{equation*} 
In laymen's terms, $f$ is holomorphic in every ``direction'' or coordinate. This
gives us a perfect segue into how the complex Jacobian matrix for a holomorphic function
$f$ is related to and defined by the Cauchy-Riemann equations:
\begin{equation*}
    J(f)(z):= \bigg(\dfrac{\partial f_i}{\partial z_j}(z)\bigg), \ 1 \leq i \leq n, \
    1 \leq j \leq m
\end{equation*}
\end{document}