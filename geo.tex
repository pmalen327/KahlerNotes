\section{Foundations of Differential Geometry}
There are a lot of important and foundational concepts in differential geometry
but we will only explore a few. We really just want to establish the differential
calculus on manifolds and perhaps some Riemannian structure. I'm assuming there
is some previous knowledge of some basic defintions like manifolds, charts, transition
maps etc.

\subsection{Algebraic Preliminaries}
To go into anymore depth, we have to take a quick detour and establish some algebraic
structure.

\begin{definition}
    Let $V$ be a vector space (over an arbitrary field $\mathbb{K}$). The $k$-fold
    product $V \times V \times \hdots \times V$ is denoted as $V^k$. A function
    $T: V^k \to \mathbb{K}$ is \ten{multilinear} if for each $i$ for $1 \leq 1 \leq k$
    we have:
    \begin{itemize}
        \item $T(v_1, \hdots, v_i + v_i', \hdots, v_k) = T(v_1, \hdots, v_i, \hdots, v_k)
        + T(v_1, \hdots, v_i', \hdots, v_k)$ 
        \item $T(v_1, \hdots, av_i, \hdots, v_k) = aT(v_1, \hdots, v_i, \hdots v_k)$
    \end{itemize}
    A multilinear function $T: V^k \to \mathbb{K}$ is a \ten{$k$-tensor} on $V$.
\end{definition}

The set of all $k$-tensors, denoted $\mathfrak{J}^k(V)$, becomes a vector space
over $\R$ if for $S,T \in \mathfrak{J}^k(V)$ and some $a \in \R$, we have:
\begin{itemize}
    \item $(S+T)(v_i) = S(v_i) + T(v_i)$
    \item $(aS)(v_i) = a \cdot S(v_i)$
\end{itemize}

You may have noticed that the dual of $V$, denoted $V^*$ is exactly $\mathfrak{J}^k(V)$.
This is more than a convenient coincidence, we will see later that this actually
lets us define other vector spaces in terms of $\mathfrak{J}^k(V)$. To connect the
spaces in $\mathfrak{J}^k(V)$ we have to introduce a general notion of a product.

\begin{definition}
    Let $S \in \mathfrak{J}^m(V)$ and $T \in \mathfrak{J}^n(V)$. We define the
    \ten{tensor product} $S \otimes T \in \mathfrak{J}^{m+n}(V)$ by:
    \begin{equation*}
        S \otimes T (v_1, \hdots, v_k, v_{m+1}, \hdots, v_{m+n}) =
        S(v_1, \hdots, v_k) \cdot T(v_{m+1}, \hdots, v_{m+n})
    \end{equation*}
\end{definition}

It is extremely important to note that the tensor product does NOT commute in general.
Changing the order of the factors will yield two entirely different spaces. However,
it does benefit from associativity and some other convenient properties:

\begin{align*}
    (S_1 + S_2) \otimes T &= S_1 \otimes T + S_2 \otimes T\\
    S \otimes (T_1 + T_2) &= S \otimes T_1 + S \otimes T_2\\
    (aS) \otimes T &= S \otimes (aT) = a(S \otimes T)\\
    (S \otimes T) \otimes U &= S \otimes (T \otimes U)
\end{align*}

To make more sense of tensors and their structure, we introduce some more exterior
algebra. We need to introduce a special type of tensor first.

\begin{definition}
    An \ten{alternating tensor} of degree $k$ on a vector space $V$ is a map
    $T: V \times \hdots \times V \to \mathbb{K}$ such that:
    \begin{itemize}
        \item $T(u_1,\hdots,u_i,\hdots,u_k) = -T(u_1,\hdots,u_j,\hdots,u_i,\hdots,u_k)$
        \item $T(\lambda_1 v_1 + \lambda_2 v_2, u_2, \hdots, u_k) = \lambda_1T(
        v_1,u_2,\hdots,u_k) + \lambda_2 T(v_2,u_2,\hdots,u_k)$
    \end{itemize}
\end{definition}

There is nothing crazy going on here but it is worth noting that the alternating
property switches the $u_i$ and $u_j$. It hints at some sort of ani-symmetric property
but not quite. Now we can look at another important ``collection'' of tensors.

\begin{definition}
    The \ten{$k$-th} exterior power $\Lambda^kV$ of a finite dimensional vector
    space $V$ is the dual space of the vector space of alternating tensors of degree
    $k$ on $V$. Elements of $\Lambda^kV$ are \ten{$k$-vectors}.
\end{definition}

Now we have this dual space of tensors but we need additional structure on it. We
introduce another product

\begin{definition}
    Given $v_1,\hdots,u_k \in V$, the \ten{exterior product} or \ten{wedge product}
    $v_1 \wedge \hdots \wedge v_k \in \Lambda^k V$ is the linear map to $\mathbb{K}$
    which, on analternating tensor $T$ takes the value:
    \begin{equation*}
        (v_1 \wedge \hdots \wedge v_k)(T) = T(v_1,\hdots,v_k)
    \end{equation*}
\end{definition}

The exterior product has a few important properties:
\begin{itemize}
    \item it is linear in each variable independently
    \item internchanging two variables changes the sign of the product
    \item two variables are the same, the product vanishes
    \item taking the product of tensors results in a tensor of a larger rank than
    either factor
\end{itemize}


One may be tempted to ask why not just use the tensor product? This wouldn't work
as the tensor product is for taking products of ``spaces" while the exterior
product allows us to take products of $k$-vectors.

\subsection{Integration on Chains}