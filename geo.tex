\section{Foundations of Differential Geometry}
There are a lot of important and foundational concepts in differential geometry
but we will only explore a few. We really just want to establish the differential
calculus on manifolds and perhaps some Riemannian structure. I'm assuming there
is some previous knowledge of some basic defintions like manifolds, charts, transition
maps etc.

\subsection{Algebraic Preliminaries}
To go into anymore depth, we have to take a quick detour and establish some algebraic
structure.

\begin{definition}
    Let $V$ be a vector space (over an arbitrary field $\mathbb{K}$). The $k$-fold
    product $V \times V \times \hdots \times V$ is denoted as $V^k$. A function
    $T: V^k \to \mathbb{K}$ is \ten{multilinear} if for each $i$ for $1 \leq 1 \leq k$
    we have:
    \begin{itemize}
        \item $T(v_1, \hdots, v_i + v_i', \hdots, v_k) = T(v_1, \hdots, v_i, \hdots, v_k)
        + T(v_1, \hdots, v_i', \hdots, v_k)$ 
        \item $T(v_1, \hdots, av_i, \hdots, v_k) = aT(v_1, \hdots, v_i, \hdots v_k)$
    \end{itemize}
    A multilinear function $T: V^k \to \mathbb{K}$ is a \ten{$k$-tensor} on $V$.
\end{definition}

The set of all $k$-tensors, denoted $\mathfrak{J}^k(V)$, becomes a vector space
over $\R$ if for $S,T \in \mathfrak{J}^k(V)$ and some $a \in \R$, we have:
\begin{itemize}
    \item $(S+T)(v_i) = S(v_i) + T(v_i)$
    \item $(aS)(v_i) = a \cdot S(v_i)$
\end{itemize}

You may have noticed that the dual of $V$, denoted $V^*$ is exactly $\mathfrak{J}^k(V)$.
This is more than a convenient coincidence, we will see later that this actually
lets us define other vector spaces in terms of $\mathfrak{J}^k(V)$. To connect the
spaces in $\mathfrak{J}^k(V)$ we have to introduce a general notion of a product.

\begin{definition}
    Let $S \in \mathfrak{J}^m(V)$ and $T \in \mathfrak{J}^n(V)$. We define the
    \ten{tensor product} $S \otimes T \in \mathfrak{J}^{m+n}(V)$ by:
    \begin{equation*}
        S \otimes T (v_1, \hdots, v_k, v_{m+1}, \hdots, v_{m+n}) =
        S(v_1, \hdots, v_k) \cdot T(v_{m+1}, \hdots, v_{m+n})
    \end{equation*}
\end{definition}

It is extremely important to note that the tensor produc does NOT commute in general.
Changing the order of the factors will yield two entirely different spaces. However,
it does benefit from associativity and some other convenient properties:

\begin{align*}
    (S_1 + S_2) \otimes T &= S_1 \otimes T + S_2 \otimes T\\
    S \otimes (T_1 + T_2) &= S \otimes T_1 + S \otimes T_2\\
    (aS) \otimes T &= S \otimes (aT) = a(S \otimes T)\\
    (S \otimes T) \otimes U &= S \otimes (T \otimes U)
\end{align*}


\subsection{Integration on Chains}