\section{The Matrix Exponential}
\subsection{The Exponential Map and Matrix Groups}
Given an $n \times n$ matrix $A$, we want to find a way to find $e^A$. We can actually
just do this with the usual power series:

\begin{equation}\label{eq:1}
    e^A = I_n + \sum_{p \ge 1} \frac{A^p}{p!} = \sum_{p \ge 0} \frac{A^p}{p!}
\end{equation}

Using an inductive proof, we can show that this is well defined. But we won't write
it out here.

\begin{boxex}{}{trigexample}
    Consider the matrix 
    \begin{equation*}
        A=
        \begin{pmatrix}
            0 & -\theta\\
            \theta & 0
        \end{pmatrix}
    \end{equation*}
    We want to find a way to express the powers $A^n$. We can factor out a $\theta$
    to see
    \begin{equation*}
        \begin{pmatrix}
            0 & -\theta\\
            \theta & 0
        \end{pmatrix}
        = \theta
        \begin{pmatrix}
            0 & -1\\
            1 & 0
        \end{pmatrix}
        \text{ and }
        \begin{pmatrix}
            0 & -\theta\\
            \theta & 0
        \end{pmatrix}^2
        =-\theta^2 I_2
    \end{equation*}
    Now, let
    \begin{equation*}
        J = 
        \begin{pmatrix}
            0 & -1\\
            1 & 0
        \end{pmatrix}
    \end{equation*}
    we have
    \begin{align*}
        A^{4n} &= \theta^{4n}I_2\\
        A^{4n+1} &= \theta^{4n+1}J\\
        A^{4n+2} &= -\theta^{4n+2}I_2
    \end{align*}
    and so on.

    So we can now express $e^A$ as a power series
    \begin{equation*}
        e^A = I_2 + \frac{\theta}{1!}J - \frac{\theta^2}{2!}I_2 \hdots
    \end{equation*}
    Writing this out we will see that we actually get the power series for cosine
    and sine, thus
    \begin{equation*}
        e^A = \cos \theta I_2 + \sin \theta J
    \end{equation*}
    or equivalently
    \begin{equation*}
        e^A = 
        \begin{pmatrix}
            \cos \theta & -\sin \theta\\
            \sin \theta & \cos \theta
        \end{pmatrix}
    \end{equation*}
    So we see $e^A$ is in fact a rotation matrix.
\end{boxex}

This is actually a general fact. If we have a skew-symmetric matrix $A$, then $e^A$
is an orthogonal matrix with determinant $1$, or $e^A \in SO_n(\F)$. In fact, EVERY
rotation matrix is of this form. To be explicit, the exponential map from the set
of skew-symmetric matrices to the set of rotation matrices is surjective. But note
that the exponential map is NOT surjective in general.

\begin{boxprop}{}{expprop}
    Let $A$ and $U$ be (real or complex) matrices and assume $U$ is invertible.
    Then
    \begin{equation*}
        e^{UAU^{-1}} = Ue^AU^{-1}
    \end{equation*}
\end{boxprop}

This is pretty obvious and its easily proven using an inductive proof. But I hate
induction so of course I will not include the proof here. 